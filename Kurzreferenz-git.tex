\documentclass[11pt, oneside]{article}   	% use "amsart" instead of "article" for AMSLaTeX format
\usepackage{geometry}                		% See geometry.pdf to learn the layout options. There are lots.
\geometry{letterpaper}                   		% ... or a4paper or a5paper or ... 
%\geometry{landscape}                		% Activate for rotated page geometry
%\usepackage[parfill]{parskip}    		% Activate to begin paragraphs with an empty line rather than an indent
\usepackage{graphicx}				% Use pdf, png, jpg, or eps§ with pdflatex; use eps in DVI mode
								% TeX will automatically convert eps --> pdf in pdflatex		
\usepackage{amssymb}

%SetFonts

%SetFonts


\title{git Kurzreferenz}
\author{manne05}
\date{}							% Activate to display a given date or no date

\begin{document}
\maketitle
\section{Installation}
Installation mit dem jeweiligen Paketmanager der Distribution bzw. direkt via https://git-scm.com/download
\subsection{Erstellen}
git init    -- erstellt ein neues Repository in einem Verzeichnis bzw. reinitialisiert ein vorhandenes Repository  \\
git add   -- Datei(en) zum Index hinzufügen \\
git commit   -- Dateien versionieren \\

\subsection{Verstehen}
Git erstellt snapshots eines Verzeichnisses, Daten (Dateien) werden immer hinzugefügt. \\
Git kennt drei Zustände \\
- modified -- eine Datei wurde verändert\\
- staged    -- Datei geaddet, aber noch nicht commited \\
- commited -- Änderungen an der Datei wurden per commit bestätigt - Änderungen werden am Repository aufgezeichnet


\end{document}  